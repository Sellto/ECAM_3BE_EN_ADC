\section{Conclusion}

\paragraph{}

Comme nous le contastons via l'application du moteur DC et comme nous l'avons exprimé précédemment, le CAN permet d'exploiter très facilement une grandeur analogique, en la traitant de manière numérique via des microcontrôleurs par exemple.

Ses utilisations son nombreuses et variées, et les grandeurs analogiques peuvent être mesurées directement dans un circuit ou captées par une antenne par exemple. Bien souvent, le CAN s'accompagne d'un Convertisseur Numérique Analogique qui permet alors de générer une nouvelle grandeur continue après voir  effectué un travail numériquement. C'est le cas du traitement des signaux en télécommunication comme nous le voyons actuellement.