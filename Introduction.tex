\section{Introduction}
\paragraph{}
Le but de ce laboratoire, est de comprendre le concept de conversion analogique/numérique au moyen d'une carte de développement dont le microprocesseur est de la famille des PIC18F.
\paragraph{}
Dans la quasi-totalité des cas, le monde physique est perçu par des signaux analogiques (Le son, la lumière...). Le traitement de ces signaux peuvent se faire de manière analogique ou numérique.  Cependant, le recours au traitement numérique offre plusieurs avantages :
\begin{itemize}
\item facilité de stockage de l'information
\item reproductibilité des traitements du signal d'entrée.
\item réalisation plus aisée de fonctionnalités de traitement complexe
\item ...
\end{itemize}
\paragraph{}
L'interface couramment utilisée pour faire le lien entre ces deux monde est le convertisseur analogique numérique (CAN) ou Analogic Digital Converter (ADC) en anglais.
%http://www.mines-stetienne.fr/~dutertre/documents/cours_convertisseurs.pdf