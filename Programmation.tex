\section{Programmation}
\label{programmation}
Ne seront présentés dans cette section que le paramétrage de l'ADC, la codage de son utilisation sera explicité dans la section \ref{notion}
\paragraph{}
Il existe deux méthodes pour configurer ces registres :

\subsection{Sans Libraire}
Il s'agit de modifier directement les bits dans les registres. Le paramétrage de l'ADC se fera sur le registres ADCON0, ADCON1 et ADCON2.

\paragraph{Tension de référence :} on configure pour que les tensions de références sont  $V_{SS}$ et $V_{DD}$


\begin{lstlisting}
ADCON1bits.VCFG0 = 0;
ADCON1bits.VCFG1 = 0;
\end{lstlisting}

\paragraph{Alignement des bits encodé} : alignement à droite.
\begin{lstlisting}
ADCON2bits.ADFM = 1;
\end{lstlisting}


\paragraph{$T_{AD}$} : paramétrage pour un $T_{AD}$ = $\frac{F_{OSC}}{16}$
\begin{lstlisting}
ADCON2bits.ADCS2 = 1;
ADCON2bits.ADCS1 = 0;
ADCON2bits.ADCS0 = 1;
\end{lstlisting}

\paragraph{$T_{ACQ}$} : paramétrage pour un $T_{ACQ}$ = 6 $T_{AD}$
\begin{lstlisting}
ADCON2bits.ACQT2 = 0;
ADCON2bits.ACQT1 = 1;
ADCON2bits.ACQT0 = 1;
\end{lstlisting}


\paragraph{Choix de l'entrée analogique :} dans cette configuration on choisi l'entrée AN0 que l'on place en "mode" analogique. 
 
\begin{lstlisting}
ADCON0bits.CHS0 = 0;
ADCON0bits.CHS1 = 0;
ADCON0bits.CHS2 = 0;
ADCON0bits.CHS3 = 0;
ADCON1bits.PCFG0 = 0;
\end{lstlisting}

\subsection{Avec la libraire}
en utilisant la libraire XC8 nous pouvont utiliser la fonction \textbf{OpenADC()} et \textbf{ SetChanADC()} pour réaliser les opérations ci-dessus.

\paragraph{}
\begin{lstlisting}
OpenADC(ADC_FOSC_16 & ADC_RIGHT_JUST & ADC_6_TAD,
            ADC_REF_VDD_VSS,
            0);

SetChanADC(ADC_CH0);
ADCON1bits.PCFG0 = 0;
\end{lstlisting}

